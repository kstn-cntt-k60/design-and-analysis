\documentclass[../report.tex]{subfiles}

\begin{document}

\subsection{Yêu cầu phi tính năng}
\begin{itemize}
    \item Yêu cầu hoạt động của hệ thống
        \begin{itemize}
            \item Hệ thống hiển thị những thông tin về sách có trong cơ sở 
                dữ liệu (tên sách, tác giả, nhà xuất bản, \dots).
            \item Hệ thống sẽ lưu lại những giao dịch mỗi khi bạn đọc thực hiện 
                việc mượn sách và xóa đi các giao dịch đã được thanh toán. 
        \end{itemize}
    \item Yêu cầu hiệu năng và bảo mật. 
        \begin{itemize}
            \item Hệ thống vẫn hoạt động ổn định và phản hồi nhanh khi có một lượng lớn truy cập ở cùng một thời điểm. 
            \item Cần sao lưu CSDL theo chu kì để tránh trường hợp bị mất dữ liệu khi xảy ra sự cố. 
        \end{itemize}
\end{itemize}
\subsection{Yêu cầu tính năng}
\begin{itemize}
    \item Tra cứu  \\
        Hệ thống quản lý thư viện cần có mộtCSDL để có thể lưu trữ các thông tin về tài liệu, 
        hồ sơ cá nhân của bạn đọc bao gồm: thông tin tài khoản và thông tin giao dịch. 
        Khi đó người dùng (bao gồm tất cả những người đang sử dụng hệ thống) có thể tìm kiếm tài liệu thông qua:
        \begin{itemize}
            \item Tên tài liệu. 
            \item Danh mục. 
            \item Tác giả. 
        \end{itemize}
        Đối với bạn đọc là thành viên của hệ thống. họ có thể chỉnh sửa thông tin 
        về tài khoản và tra cứu những giao dịch hiện chưa hoàn tất thanh toán.

    \item Cập nhật tài liệu
        \begin{itemize}
            \item Khi có tài liệu mới được thêm vào kho tài liệu của thư viện, quản lý sẽ thêm thông tin của sách vào trong CSDL.
            \item Khi bạn đọc mượn tài liệu, nhưng sau một thời gian vì một số lý do nào đó cuốn sách bị mất. 
                Lúc đó bạn đọc sẽ thông báo mất sách cho hệ thống, hệ thống xác nhận lại thông tin 
                và xóa đi thông tin của tài liệu trong CSDL và yêu cầu bạn đọc bồi thường.
        \end{itemize}

    \item Xử lý giao dịch \\
        Mỗi khi bạn đọc mượn tài liệu từ thư viện, hệ thống sẽ yêu cầu kê khai 
        thông tin bạn đọc và thông tin tài liệu mượn vào phiếu hóa đơn giao dịch. 
        Phiếu này sau đó được lưu lại trong CSDL và được xóa đi khi tài khoản của bạn đọc bị xóa.
\end{itemize}

\end{document}
