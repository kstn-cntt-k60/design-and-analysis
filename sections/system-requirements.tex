\documentclass[../report.tex]{subfiles}

\begin{document}

Hệ thống được xây dựng dựa trên mô hình hoạt động của hệ thống thư viện trường Đại học Bách Khoa Hà Nội.	 
\begin{itemize}
    \item Mức độ cần thiết của dự án: Hệ thống được tạo ra với mục đích cung cấp cho người đọc nhiều tiện ích trong 
        việc tra cứu thông tin tài liệu. Đồng thời, hỗ trợ cho công việc quản lý thư viện được dễ dàng hơn.	 
    \item Hệ thống cần cung cấp những gì?	 \\
        Hệ thống sẽ phải phục vụ cho hai tác nhân chính đó là bạn đọc và người quản lý thư viện.
        \begin{itemize}
            \item Đối với bạn đọc, họ có thể truy cập hệ thống từ website. 
            Trên đó bạn đọc có thể tra cứu thông tin về tài liệu hiện có trong thư 
            viện và các thông tin liên quan tới cá nhân như thông tin tài khoản, 
            thông tin những giao dịch chưa hoàn tất thanh toán, thông tin tài liệu đang dược mượn.
            \item Đối với người quản lý thư viện, hệ thống cung cấp cho họ các chức năng:
                \begin{itemize}
                    \item Quản lý thông tin nhập, xuất sách.
                    \item Quản lý các giao dịch của bạn đọc.
                \end{itemize}
        \end{itemize}
    \item Tiềm năng của dự án \\
        Chúng tôi kì vọng hệ thống sẽ giúp cho bạn đọc tra cứu tài liệu một cách dễ dàng hơn, 
        việc cập nhập thông tin về những thay đổi của thư viện đến bạn đọc một cách nhanh chóng. 
        Đồng thời, hệ thống sẽ giảm bớt được khối lượng công việc của người quản lý trong việc 
        vận hành thư viện và quảng bá được hình ảnh của thư viện tới nhiều bạn đọc hơn nữa.
\end{itemize}

\end{document}
